%% This file is generated by Jinja2
\documentclass[border=1pt]{standalone}
\usepackage{tikz}
\usetikzlibrary{arrows,positioning}

%%%<
\usepackage{verbatim}
\usepackage[active,tightpage]{preview}
\PreviewEnvironment{tikzpicture}
\setlength\PreviewBorder{1pt}%
%%%>

\begin{comment}
:title: Simple cycle
:author: Long Gong

A template for drawing a cycle. Note that, this template is highly
"inspired" by http://www.texample.net/tikz/examples/cycle/

In some ways, this TeX script works as the "model" of our application
for visualizing a cycle.

Programmed in TikZ by Long Gong. Templating language is Jinja2,
templaing syntax is the default setting of Jinja2.
\end{comment}

\begin{document}

\begin{tikzpicture}[
vertex/.style={circle, thick, inner sep=1pt, minimum size=1.8em, draw=black, fill=white},
edge/.style={thick},
edge-start-with-arrow/.style={edge, <-, >=latex'},
edge-end-with-arrow/.style={edge, ->, >=latex'},
dummy/.style={draw=none,fill=none,inner sep=1pt}
]

\def\radius{ 18.4em }
\def\margin{ 3  } %% in terms of angle


%% draw edges on the cycle
\draw[edge,('red', 'green')] ({0 + \margin}:\radius) arc ({0 + \margin}:{22 - \margin}:\radius) node[dummy, pos=0.5, swap, auto]{\textcolor{black}{$5$}};
\draw[edge,('red', 'green')] ({22 + \margin}:\radius) arc ({22 + \margin}:{45 - \margin}:\radius) node[dummy, pos=0.5, swap, auto]{\textcolor{black}{$5$}};
\draw[edge,('red', 'green')] ({45 + \margin}:\radius) arc ({45 + \margin}:{67 - \margin}:\radius) node[dummy, pos=0.5, swap, auto]{\textcolor{black}{$6$}};
\draw[edge,('red', 'green')] ({67 + \margin}:\radius) arc ({67 + \margin}:{90 - \margin}:\radius) node[dummy, pos=0.5, swap, auto]{\textcolor{black}{$8$}};
\draw[edge,('red', 'green')] ({90 + \margin}:\radius) arc ({90 + \margin}:{112 - \margin}:\radius) node[dummy, pos=0.5, swap, auto]{\textcolor{black}{$6$}};
\draw[edge,('red', 'green')] ({112 + \margin}:\radius) arc ({112 + \margin}:{135 - \margin}:\radius) node[dummy, pos=0.5, swap, auto]{\textcolor{black}{$4$}};
\draw[edge,('red', 'green')] ({135 + \margin}:\radius) arc ({135 + \margin}:{157 - \margin}:\radius) node[dummy, pos=0.5, swap, auto]{\textcolor{black}{$2$}};
\draw[edge,('red', 'green')] ({157 + \margin}:\radius) arc ({157 + \margin}:{180 - \margin}:\radius) node[dummy, pos=0.5, swap, auto]{\textcolor{black}{$1$}};
\draw[edge,('red', 'green')] ({180 + \margin}:\radius) arc ({180 + \margin}:{202 - \margin}:\radius) node[dummy, pos=0.5, swap, auto]{\textcolor{black}{$2$}};
\draw[edge,('red', 'green')] ({202 + \margin}:\radius) arc ({202 + \margin}:{225 - \margin}:\radius) node[dummy, pos=0.5, swap, auto]{\textcolor{black}{$8$}};
\draw[edge,('red', 'green')] ({225 + \margin}:\radius) arc ({225 + \margin}:{247 - \margin}:\radius) node[dummy, pos=0.5, swap, auto]{\textcolor{black}{$5$}};
\draw[edge,('red', 'green')] ({247 + \margin}:\radius) arc ({247 + \margin}:{270 - \margin}:\radius) node[dummy, pos=0.5, swap, auto]{\textcolor{black}{$4$}};
\draw[edge,('red', 'green')] ({270 + \margin}:\radius) arc ({270 + \margin}:{292 - \margin}:\radius) node[dummy, pos=0.5, swap, auto]{\textcolor{black}{$7$}};
\draw[edge,('red', 'green')] ({292 + \margin}:\radius) arc ({292 + \margin}:{315 - \margin}:\radius) node[dummy, pos=0.5, swap, auto]{\textcolor{black}{$7$}};
\draw[edge,('red', 'green')] ({315 + \margin}:\radius) arc ({315 + \margin}:{337 - \margin}:\radius) node[dummy, pos=0.5, swap, auto]{\textcolor{black}{$6$}};
\draw[edge,('red', 'green')] ({337 + \margin}:\radius) arc ({337 + \margin}:{360 - \margin}:\radius) node[dummy, pos=0.5, swap, auto]{\textcolor{black}{$5$}};

%% draw nodes on the cycle
%% NOTE, here we choose to place nodes after edges, because
%% we want to "cover" the parts of edges getting inside the
%% vertices. We are looking for better solutions for fixing
%% this issue.
\node[vertex] (0) at (0:\radius) {$I_1$};
\node[vertex] (12) at (22:\radius) {$O_5$};
\node[vertex] (6) at (45:\radius) {$I_7$};
\node[vertex] (8) at (67:\radius) {$O_1$};
\node[vertex] (7) at (90:\radius) {$I_8$};
\node[vertex] (14) at (112:\radius) {$O_7$};
\node[vertex] (5) at (135:\radius) {$I_6$};
\node[vertex] (10) at (157:\radius) {$O_3$};
\node[vertex] (1) at (180:\radius) {$I_2$};
\node[vertex] (9) at (202:\radius) {$O_2$};
\node[vertex] (3) at (225:\radius) {$I_4$};
\node[vertex] (13) at (247:\radius) {$O_6$};
\node[vertex] (2) at (270:\radius) {$I_3$};
\node[vertex] (11) at (292:\radius) {$O_4$};
\node[vertex] (4) at (315:\radius) {$I_5$};
\node[vertex] (15) at (337:\radius) {$O_8$};

\end{tikzpicture}

\end{document}