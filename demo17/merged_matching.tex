%% This file is generated by Jinja2
% Template for bipartite weighted matching
% Author: Long Gong
\documentclass[border=2pt]{standalone}
%%%<
\usepackage{verbatim}
%%%>
\begin{comment}
:Title: Template for bipartite weighted matching
:Author: Long Gong

A template for bipartite weighted matching. 

In some ways, this TeX script works as the "model" of our application 
for visualizing a weighted bipartite matching. 


Programmed in TikZ by Long Gong. Templating language is Jinja2, 
templaing syntax is the default setting of Jinja2.
\end{comment}


\usepackage{tikz}
\usetikzlibrary{calc,positioning}

\begin{document}
\begin{tikzpicture}[
vertex/.style={circle, draw, inner sep=4pt, thick},
edge/.style={thick},
weight/.style={rectangle, draw, inner sep=2pt, minimum width=20pt},
info/.style={draw=none,fill=none,inner sep=0pt}]

%% local variables
\def \margin{48pt}
\def \hm {100pt}
\def \vm {20pt}
\def \NUMOFVERTICES {8} 

%% place all input vertices
\foreach \s in {1,...,\NUMOFVERTICES}
      \node[vertex,label=left:$\s$] (I-\s) at (0,{- (\s - 1) * \vm}) {};

%% place all output vertices
\foreach \s in {1,...,\NUMOFVERTICES}
      \node[vertex,label=right:$\s$] (O-\s) at (\hm,{- (\s - 1) * \vm}) {};

\node[info] (I) at (0,{- (\NUMOFVERTICES - 1) * \vm}) {};
\node[info] (O) at (\hm,{- (\NUMOFVERTICES - 1) * \vm}) {};

%% place other ifnormation
\node[info] (in) at (0,\vm) {\bf Input};
\node[info] (out) at (\hm, \vm){\bf Output};
\node[info] (weight) at ({\hm+\margin}, \vm) {\bf Weight};

%% matching information
% ==========================================
% 1 3 2
% 2 1 1
% 3 7 6
% 4 5 1
% 5 6 8
% 6 2 2
% 7 8 6
% 8 4 7
% ==========================================

%% place all weight right of output vertices
%% previous version failed to change to place the weight near o instead of i
\node[weight] (W-3) at ({\hm+\margin},{- (2) * \vm}) {$2$};
%% previous version failed to change to place the weight near o instead of i
\node[weight] (W-1) at ({\hm+\margin},{- (0) * \vm}) {$1$};
%% previous version failed to change to place the weight near o instead of i
\node[weight] (W-7) at ({\hm+\margin},{- (6) * \vm}) {$6$};
%% previous version failed to change to place the weight near o instead of i
\node[weight] (W-5) at ({\hm+\margin},{- (4) * \vm}) {$1$};
%% previous version failed to change to place the weight near o instead of i
\node[weight] (W-6) at ({\hm+\margin},{- (5) * \vm}) {$8$};
%% previous version failed to change to place the weight near o instead of i
\node[weight] (W-2) at ({\hm+\margin},{- (1) * \vm}) {$2$};
%% previous version failed to change to place the weight near o instead of i
\node[weight] (W-8) at ({\hm+\margin},{- (7) * \vm}) {$6$};
%% previous version failed to change to place the weight near o instead of i
\node[weight] (W-4) at ({\hm+\margin},{- (3) * \vm}) {$7$};

%% place all edges
\foreach \i/\o/\c in {1/3/g, 2/1/r, 3/7/g, 4/5/g, 5/6/g, 6/2/g, 7/8/g, 8/4/g}
      \draw[edge,\c] (I-\i) -- (O-\o);

\end{tikzpicture}
\end{document}