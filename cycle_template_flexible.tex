\documentclass[border=1pt]{standalone}
\usepackage{tikz}
\usetikzlibrary{arrows,positioning}

%%%<
\usepackage{verbatim}
\usepackage[active,tightpage]{preview}
\PreviewEnvironment{tikzpicture}
\setlength\PreviewBorder{1pt}%
%%%>

\begin{comment}
:title: Simple cycle
:author: Long Gong

A template for drawing a cycle. Note that, this template is highly
"inspired" by http://www.texample.net/tikz/examples/cycle/

In some ways, this TeX script works as the "model" of our application
for visualizing a cycle.

Programmed in TikZ by Long Gong. Templating language is Jinja2,
templaing syntax is the default setting of Jinja2.
\end{comment}

\begin{document}

\begin{tikzpicture}[
vertex/.style={circle, thick, inner sep=1pt, minimum size={{ node_size }}em, draw=black, fill=none},
edge/.style={thick},
edge-start-with-arrow/.style={edge, <-, >=latex'},
edge-end-with-arrow/.style={edge, ->, >=latex'},
dummy/.style={draw=none,fill=none,inner sep=1pt}
]

\def\radius{{ '{' }} {{ radius }}em {{ '}' }}
\def\margin{{ '{' }} {{ margin }}  {{ '}' }} %% in terms of angle

%% draw nodes on the cycle

\node[vertex] ({{ node.identity }}) at ({{ node.position }}:\radius) {{ '{$' }}{{ node.label }}{{ '$}' }};


%% draw edges on the cycle

\draw[edge-end-with-arrowedge-start-with-arrowedge,{{edge.color}}] ({{ edge.start }}:\radius) arc ({{ edge.start }}:{{ edge.end }}:\radius) node[dummy, pos=0.5, swap, auto]{{'{$'}}{{edge.weight}}{{'$}'}};


\end{tikzpicture}

\end{document}