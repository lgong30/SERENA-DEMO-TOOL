\documentclass[border=1pt]{standalone}
\usepackage{tikz}
\usetikzlibrary{arrows,positioning} 

%%%<
\usepackage{verbatim}
\usepackage[active,tightpage]{preview}
\PreviewEnvironment{tikzpicture}
\setlength\PreviewBorder{1pt}%
%%%>
                             
\begin{comment}
:title: Simple cycle
:author: Long Gong

A template for drawing a cycle. Note that, this template is highly 
"inspired" by http://www.texample.net/tikz/examples/cycle/

In some ways, this TeX script works as the "model" of our application 
for visualizing a cycle. 

Programmed in TikZ by Long Gong. Templating language is Jinja2, 
templaing syntax is the default setting of Jinja2.
\end{comment}

\begin{document}

\begin{tikzpicture}[
vertex/.style={circle, thick, inner sep=1pt, minimum size=1.8em, draw=black, fill=none},
edge/.style={thick},
edge-start-with-arrow/.style={edge, <-, >=latex'},
edge-end-with-arrow/.style={edge, ->, >=latex'}
]

\def\radius{ 2.9em }
% \def\margin{   } %% in terms of angle

%% draw nodes on the cycle
\node[vertex] (11) at (0:\radius) {$O_4$};
\node[vertex] (4) at (180:\radius) {$I_5$};

%% draw edges on the cycle
\draw[edge,green] (19:\radius) arc (19:161:\radius) node[midway, auto, sloped]{$4$};
\draw[edge,red] (199:\radius) arc (199:341:\radius) node[midway, auto, sloped]{$6$};

\end{tikzpicture}

\end{document}